\documentclass[../../fsbok.tex]{subfiles} 
\begin{document}
\begin{instruks}{Instruks for bruk av Tåkeheimen}{}{}

    \begin{instruksledd}{Intensjon og formål}
Instruksens intensjon er å klargjøre bruken av og ansvaret for lokalene ved Kuppelen nordre side(\textquotedblleft Tåkeheimen"). Tåkeheimen skal brukes til møtevirksomhet og annet kontorarbeid. 

    \end{instruksledd}

    \begin{instruksledd}{Eierskap og bruk av lokaler}
        \begin{enumerate}
		\item \textbf{Innsikten}
			
Innsikten er et lokale til møtevirksomhet for FSUG, gjenger og andre som har et behov for møtelokale. Innsikten disponeres primært av Gjengsekretariatet og all bruk skal avklares med dem først.
		
\item \textbf{Klubba}
			
Klubba er lokalet til følgende tre styrer: 
			\begin{enumerate}
			\item Styret ved Studentersamfundet i Trondhjem
			\item UKA-styret
			\item ISFiT-styret
			\end{enumerate}

			 Lokalet disponeres primært av styret til Studentersamfundet, og skal i samråd med styrene til ISFiT og UKA fordele hvilke tider hvert styre skal bruke lokalet. Ved behov, og mulighet, kan andre også bruke lokalet, men da først etter avtale med styret til Studentersamfundet.
		
\item \textbf{Øvrigheten}

		Øvrigheten er et arbeidslokale som kan brukes etter avtale med styret til Studentersamfundet eller Gjengsekretariatet.

\item \textbf{Klang}

		Klang er et felles kjøkken for alle som bruker Tåkeheimen.

        \end{enumerate}
    \end{instruksledd}

    \begin{instruksledd}{Prioriteringsnormer}
	\begin{enumerate}
		\item Innsikten
			\begin{enumerate}
				\item Gjengsekretariatet
				\item FSUG
				\item Andre
			\end{enumerate}
		\item Klubba
			\begin{enumerate}
				\item Styret ved Studentersamfundet
				\item Styrene til UKA og ISFiT
				\item Andre
			\end{enumerate}
	\end{enumerate}
    \end{instruksledd}

    \begin{instruksledd}{Renhold og orden}
	

		Alle rom skal ryddes etter bruk. Det skal ikke benyttes sko på Tåkeheimen.

	

    \end{instruksledd}
	
\end{instruks}
\end{document}