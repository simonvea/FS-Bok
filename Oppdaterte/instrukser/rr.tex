\documentclass[../../fsbok.tex]{subfiles} 
\begin{document}
\begin{instruks*}{Instruks for Radio Revolt}

    \begin{instruksledd}{Formål}
        \begin{enumerate}
            \item Radio Revolt, studentradioen i Trondheim skal drive radiosendinger for studentene
                i Trondheim og byens befolkning.
        \end{enumerate}
    \end{instruksledd}

    \begin{instruksledd}{Sammensetning}
        \begin{enumerate}
            \item Radio Revolt består av 85 medlemmer, hvorav 15 er funksjonærer. Nye medlemmer tas
                opp en gang i semesteret, med mulighet for suppleringsopptak utover dette. Bare medlemmer av
                Studentersamfundet kan tas opp i Radio Revolt.
            \item Et medlem av Radio Revolt er aktivt i minimum 2 år. Et medlem kan innvilges permisjon av
                redaksjonen og kan også fritas fra sitt gjengmedlemskap.
            \item Gjengsjef velges av Radio Revolts allmøte og velges for ett år av gangen. Gjengsjef har ansvar for gjengen internt på
Samfundet og har det øvrige personal- og driftsansvar.
            \item Radio Revolt opprettholder normale sendinger under UKA.
        \end{enumerate}
    \end{instruksledd}
    
    
    \begin{instruksledd}{Ansvarsområde og plikter}
        \begin{enumerate}   
            \item  Medlemmer i Radio Revolt plikter å kjenne innholdet av og overholde følgende
                instrukser på Huset:
                \begin{enumerate}
                    \item Generell gjenginstruks
			\item Hybelinstruks
                \end{enumerate}
            \item Radio Revolt disponerer og har ansvaret for følgende rom med tilhørende utstyr:
                \begin{enumerate}
                    \item Blæsten
                \end{enumerate}
            \item Radio Revolt plikter å utføre arbeid i henhold til punkt 1. i tidsrommene
                medio august til medio desember og medio januar til medio juni. Dato for arbeid og andre tidsfrister
                settes av redaktøren.
            \item Gjengsjef er ansvarlig for Radio Revolts utstyr på Samfundet, for at dette blir
fagmessig betjent, og at Radio Revolts utstyr blir forsvarlig sikret mot tyveri og skade.
Gjengsjefen er ansvarlig for at brannvernforskrifter og sikringsbestemmelser blir
overholdt på Radio Revolts område.
        \end{enumerate}
    \end{instruksledd}


    \begin{instruksledd}{Økonomi}
        \begin{enumerate}
            \item Gjengsjefen er ansvarlig for den biten av økonomien som ikke er knyttet til
                redaksjonell drift avRadio Revolt.
        \end{enumerate}
    \end{instruksledd}
    
    \begin{instruksledd}{Endring av instruksen}
        \begin{enumerate}
            \item Endringer av denne instruksen skal skje i samsvar med Studentersamfundets lover og
i samråd med Radio Revolts medlemmer. Alle endringer skal forelegges Finansstyret
til godkjenning.
        \end{enumerate}
    \end{instruksledd}
    
    \begin{instruksledd}{Formidling}
        \begin{enumerate}
            \item Gjengsjefen plikter å gjøre nye medlemmer av Radio Revolt kjent med denne
                instruks.
        \end{enumerate}
    \end{instruksledd}

\end{instruks*}
\end{document}

