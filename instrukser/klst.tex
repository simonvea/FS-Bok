\documentclass[../fsbok.tex]{subfiles} 
\begin{document}
\begin{instruks}{Instruks for Klubbstyret}{20. november 2011}{13. desember 2011}

    \begin{instruksledd}{Formål}
        \begin{enumerate}
            \item Klubbstyret (KLST) har ansvaret for å planlegge og arrangere fredagskveldene på
                Studentersamfundet i
                Trondhjem og skal bidra til å gjøre Studentersamfundet til et senter for kulturelle arrangementer og
                aktiviteter, samt tilby et utelivstilbud til Studentersamfundets medlemmer. KLST er også ansvarlig
                for
                husorden og drifter Studentersamfundets bibliotek.
        \end{enumerate}
    \end{instruksledd}

    \begin{instruksledd}{Sammensetning}
        \begin{enumerate}
            \item KLST består av inntil 22 faste medlemmer, hvorav 8 er funksjonærer.
            \item En funksjonær i KLST er aktiv i 2 år og har etter dette mulighet for å beholde
                funksjonærstatusen som
                pangsjonist.
            \item Husmann er gjengsjef i KLST og velges av Klubbstyrets medlemmer, funksjonærtiden
                er 1 år. Bare den som
                er/har vært medlem i Klubbstyret kan bli Husmann.
            \item KLST opprettholder ikke virksomheten under UKA.
        \end{enumerate}
    \end{instruksledd}

    \begin{instruksledd}{Ansvarsområde og plikter}
        \begin{enumerate}
            \item  Medlemmer i KLST plikter å kjenne innholdet av og overholde følgende instrukser
                på Huset:
                \begin{enumerate}
                    \item Generell gjenginstruks
                \end{enumerate}
            \item KLST disponerer og har ansvaret for følgende rom med tilhørende utstyr:
                \begin{enumerate}
                    \item Klubbstyrets hybel
                    \item Bibliotek med tilhørende kontor, lager og kjøkken
                    \item Billettbodene ved inngang (i samarbeids med Lørdagskomiteen)
                \end{enumerate}
        \end{enumerate}
    \end{instruksledd}

    \begin{instruksledd}{Formidling}
        \begin{enumerate}
            \item Gjengsjefen plikter å gjøre nye medlemmer av KLST kjent med denne instruks.
        \end{enumerate}
    \end{instruksledd}


\end{instruks}
\end{document}

