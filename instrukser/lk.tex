\documentclass[../fsbok.tex]{subfiles} 
\begin{document}
\begin{instruks*}{Instruks for Lørdagskomit\'een}

    \begin{instruksledd}{Formål}
        \begin{enumerate}
            \item Lørdagskomit\'een (LØK) har ansvaret for planlegging og gjennomføring av lørdagenes
                kultur- og
                utelivstilbud på Studentersamfundet i Trondhjem, med unntak av Samfundsmøtet, og skal i samarbeid
                med
                Styret videreutvikle Studentersamfundets tilbud til medlemmer og andre besøkende.
        \end{enumerate}
    \end{instruksledd}

    \begin{instruksledd}{Sammensetning}
        \begin{enumerate}
            \item LØK består av inntil 16 faste medlemmer, hvorav 8 er funksjonærer.
            \item En funksjonær i LØK er aktiv i 2 år og har etter dette mulighet for å beholde
                funksjonærstatusen som
                pangsjonist. Et gjengmedlem er aktivt i 1 år.
            \item LØK opprettholder ikke virksomheten under UKA.
        \end{enumerate}
    \end{instruksledd}

    \begin{instruksledd}{Ansvarsområde og plikter}
        \begin{enumerate}
            \item  Medlemmer i LØK plikter å kjenne innholdet av og overholde følgende instrukser på
                Huset:
                \begin{enumerate}
                    \item Generell gjenginstruks.
                \end{enumerate}
            \item LØK disponerer og har ansvaret for følgende rom med tilhørende utstyr:
                \begin{enumerate}
                    \item Styrets hybel og kontor Friheten (i samarbeid med Styret)
                    \item Billettboden ved inngang (i samarbeid med Klubbstyret)
                    \item LØK har mulighet til å benytte Styrets hybel på midlertidig basis, etter
                        nærmere avtale med Styret.
                \end{enumerate}
        \end{enumerate}
    \end{instruksledd}

    \begin{instruksledd}{Formidling}
        \begin{enumerate}
            \item Gjengsjefen plikter å gjøre nye medlemmer av LØK kjent med denne instruks.
        \end{enumerate}
    \end{instruksledd}

\end{instruks*}

\end{document}
