\documentclass[../fsbok.tex]{subfiles} 
\begin{document}
\begin{instruks}{Presseinstruks}{20. november 2011}{13. desember 2011}


    \begin{instruksledd}{Formål}

        Formålet med denne instruksen er å gi de frivillige på Samfundet retningslinjer de kan følge når det kommer til pressehåndtering.

    \end{instruksledd}


    \begin{instruksledd}{Uttalerett}

       Leder for Studentersamfundet og Finansstyrets leder er de eneste som kan uttale seg på vegne av Studentersamfundet.


I saker som gjelder administrative og driftsmessige forhold på Samfundet skal Daglig Leder kunne gi uttalelser.


Alle personer nevnt over kan viderebefordre sin uttalerett til utvalgte personer, enten fra sak til sak eller på generell basis. Dette må gjøres skriftlig.

    \end{instruksledd}


    \begin{instruksledd}{Gjengenes uttalerett}

Gjengsjefene har rett til å uttale seg på forhold som gjelder sin gjeng spesielt.


Gjengsjefen kan viderebefordre sin uttalerett til utvalgte personer, enten fra sak til sak eller på generell basis. Dette må gjøres skriftlig.

    \end{instruksledd}

\begin{instruksledd}{Bruk av Pressepass}

Journalister på jobb på Samfundet skal ha pressepass synlig hele den tiden de er på Samfundet i journalistisk øyemed. Pressepasset skal være Samfundets eget, og skal tildeles etter avtale med en presseansvarlig. 


Presseansvarlige i den gjengen som er ansvarlige for arrangementet, har ansvar for at pressepasset blir utdelt og at det blir informert om presseinstruksen. Ved Samfundsmøter eller utenom arrangement er det Styrets presseansvarlig sitt ansvar.


Presseansvarlig i Styret skal sørge for at det til enhver tid er pressepass tilgjengelig.

    \end{instruksledd}


    \begin{instruksledd}{Mediegjengene}

Mediegjengene som har tilhørighet på Samfundet kan bruke eget pressepass på jobb i journalistisk øyemed. Gjengen er selv ansvarlig for å produsere disse. Har de ikke pressepass skal de få av presseansvarlige jfr punkt 4.


Representanter fra mediegjengene skal på lik linje med annen presse be om tillatelse om å delta på arrangement, samt informere om at de er tilstede så lenge de er der i journalistisk øyemed.

    \end{instruksledd}

  \begin{instruksledd}{Brudd på Instruksen}

Journalister som bryter noen av de overnevnte instruksene, kan uten advarsel bortvises fra Samfundet av vaktene, representanter fra den arrangerende gjengen, representanter fra Styret eller Administrasjonen.

\end{instruksledd}

\end{instruks}
\end{document}